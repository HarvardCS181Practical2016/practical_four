\documentclass{article}
\usepackage{graphicx}
\usepackage{float}

\begin{document}

\title{CS181 Spring 2016 Practical 4: Reinforcement Learning | Team EXT3}
\author{Robert J. Johnson | Dinesh Malav | Matthew McKenna}


\maketitle

\begin{abstract}
Swingy Monkey is a very basic game written in Python. Our project focused on using this game as a simplified context to develop and gain exposure to reinforcement learning algorithms. Our study showed that after x amount of epochs, our score was able to reach y, a vast improvement over random movements from the monkey.
\end{abstract}
\section{Technical Approach}
Our main focus was an implementation of  a particular reinforcement learning algorithm, Q-Learning, in Python in order to develop a policy for the monkey to navigate the course. In basic terms, Q-Learning is an unsupervised learning algorithm that essentially stores value information regarding a particular approach to an outcome-based event in a matrix and uses this matrix to make further decisions in later epochs.\\\\
Of particular importance is the algorithm's ability to infer how the gravity changes during each epoch of the game. Since the pattern is random, we had to quickly find a way to deduce what the rate was in order to determine our policy for navigating the trees. 
\section{Results}
Our monkey was able to be trained to reach a score of X after Y epochs. 

\section{Discussion}

All code for this project can be found at: $$https://github.com/HarvardCS181Practical2016$$.





\end{document}